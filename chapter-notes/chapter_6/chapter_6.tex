\documentclass[12pt,notitlepage]{article}

% for margins
\usepackage[a4paper,left=3cm,right=2cm,top=2.5cm,bottom=2.5cm]{geometry}
% for citations
\usepackage{apacite}
% for hyperlinks
\usepackage{hyperref}
% for maths symbols like the real numbers
\usepackage{amssymb}


\begin{document}
\bibliographystyle{apacite}

\title{\Large{\textbf{Chapter 6}}}
\date{October 18, 2020}
\author{Simon Ward-Jones\\simonwardjones16@gmail.com}

\maketitle
% \tableofcontents
\href{https://d2l.ai/chapter_convolutional-neural-networks/index.html}{Link to chapter}

\section{Convolutional Neural Networks}
\begin{itemize}
    \item Convolutional Neural Networks
    \item Translation invariance in images implies that all patches of an image will be treated in the same manner.
    \item Locality means that only a small neighborhood of pixels will be used to compute the corresponding hidden representations.
    \item In image processing, convolutional layers typically require many fewer parameters than fully-connected layers.
    \item The core computation of a two-dimensional convolutional layer is a two-dimensional cross-correlation operation. In its simplest form, this performs a cross-correlation operation on the two-dimensional input data and the kernel, and then adds a bias.
\end{itemize}

\section{Padding and Stride}
\begin{itemize}
    \item Padding can increase the height and width of the output. This is often used to give the output the same height and width as the input.
    \item The stride can reduce the resolution of the output, for example reducing the height and width of the output to only 1/n  of the height and width of the input ( n is an integer greater than  1 ).
\end{itemize}
\vfill
\bibliography{../References}
\nocite{zhang2020dive}
\end{document}

