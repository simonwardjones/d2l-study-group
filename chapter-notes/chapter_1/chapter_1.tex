\documentclass[12pt,notitlepage]{article}

% for margins
\usepackage[a4paper,left=3cm,right=2cm,top=2.5cm,bottom=2.5cm]{geometry}
% for citations
\usepackage{apacite}
% for hyperlinks
\usepackage{hyperref}
% for maths symbols like the real numbers
\usepackage{amssymb}


\begin{document}
\bibliographystyle{apacite}

\title{\Large{\textbf{Chapter 1}}}
\date{July 31, 2020}
\author{Simon Ward-Jones}

\maketitle
% \tableofcontents
\href{https://d2l.ai/chapter_introduction/index.html}{Jump to chapter}

\section{Motivating example}
\begin{itemize}
    \item How to identify ``Hey Siri'' without machine learning?
    \item The (supervised) ML process is broadly as follows:
          \subitem Pick a model and initialise parameters then repeat:
          \begin{itemize}
              \item Measure how good the model is against data
              \item Change params (to ideally improve the error)
          \end{itemize}
    \item In ML we don't design the solution explicitly
    \item Deep models are deep in precisely the sense that they learn many layers of computation
\end{itemize}

\section{The Key Components: Data, Models, and Algorithms}
\begin{itemize}
    \item \textbf{Data} - what we can learn from
          \begin{itemize}
              \item examples, targets and features
              \item dimensionality - fixed length Vs variable-length
              \item garbage in, garbage out
          \end{itemize}
    \item \textbf{Model} - how to transform the data.
    \item \textbf{Loss function} - quantifies the badness of our model.
          \begin{itemize}
              \item minimize the loss to give the best parameters
              \item training and testing error and overfitting
              \item $\mathbf{L}_1$, $\mathbf{L}_2$ and cross entropy
          \end{itemize}
    \item \textbf{Algorithm} - how to adjust the model’s parameters to minimize the loss (gradient descent)
\end{itemize}

\section{Models}

\begin{itemize}
    \item \textbf{Supervised}
          \begin{itemize}
              \item \textbf{regression} - real number target e.g. predicting house prices
              \item \textbf{classification} - discrete number of targets e.g. predicting heart attacks $y \in \{0,1\}$
              \item find $f_{\theta}$ such that $f_{\theta}(\textbf{x}_i) = \hat{y}_i\sim y_i$ given data $\{\textbf{x}_i,y_i\}^n_{i=1} $
              \item for classification we are (often) interested in $P(y|\textbf{x})$
              \item \textbf{tagging} - extending classification to multi-label classification (dog + cat, AWS, medical journal tagging)
              \item \textbf{search and ranking} - ordered relevant subset retrieval
              \item \textbf{recommender systems} - personalised recommendation subset retrieval
              \item \textbf{sequence learning} - using sequential nature of input features
                    \begin{itemize}
                        \item tagging and parsing - annotating a text sequence with attributes (e.g. named entities)
                        \item automatic speech recognition - from audio waves to text
                        \item text to speech - from text to audio waves
                        \item machine translation
                    \end{itemize}
          \end{itemize}
    \item \textbf{Unsupervised}
          \begin{itemize}
              \item \textbf{clustering} - grouping "similar" samples
              \item \textbf{subspace estimation} - dimensionality reduction
              \item \textbf{representation learning} - representing in $\mathbb{R}^n$
              \item \textbf{probabilistic graphical models} - defining causality
              \item \textbf{generative adversarial networks (GANs)} - synthesize data
          \end{itemize}
    \item \textbf{Interacting with an Environment}
          \begin{itemize}
              \item offline learning doesn't interact with environment
              \item intelligent agents are trained to make actions
              \item \textbf{Reinforcement learning} - e.g. self driving car. The agent interacts
                    with the environment via actions and observations over a period of time steps. The
                    agent receives a reward after each action (defining this reward is a problem
                    itself). The policy (i.e. what action to take in a certain scenario) is learnt
                    balancing exploration and exploitation.
          \end{itemize}
\end{itemize}

\section{Exercises}
\begin{enumerate}
    \item \textit{Which parts of code that you are currently writing could be “learned”, i.e.,
              improved by learning and automatically determining design choices that are made in your code? Does your code include heuristic design choices?}

          A good starting place would be the syntax; a large set of labelled data, e.g. correct or incorrect syntax could be collected
          and used to train a classifier.
    \item \textit{Which problems that you encounter have many examples for how to solve them, yet no specific way to automate them? These may be prime candidates for using deep learning.}

          answer
    \item \textit{Viewing the development of artificial intelligence as a new industrial revolution, what is the relationship between algorithms and data? Is it similar to steam engines and coal (what is the fundamental difference)?}

          answer
    \item \textit{Where else can you apply the end-to-end training approach? Physics? Engineering? Econometrics?}
          answer
\end{enumerate}


\vfill
\bibliography{../References}
\nocite{LeCun2015}
\nocite{zhang2020dive}
\end{document}

