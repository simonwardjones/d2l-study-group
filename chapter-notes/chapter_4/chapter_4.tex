\documentclass[12pt,notitlepage]{article}

% for margins
\usepackage[a4paper,left=3cm,right=2cm,top=2.5cm,bottom=2.5cm]{geometry}
% for citations
\usepackage{apacite}
% for hyperlinks
\usepackage{hyperref}
% for maths symbols like the real numbers
\usepackage{amssymb}


\begin{document}
\bibliographystyle{apacite}

\title{\Large{\textbf{Chapter 4}}}
\date{September 6, 2020}
\author{Simon Ward-Jones\\simonwardjones16@gmail.com}

\maketitle
% \tableofcontents
\href{https://d2l.ai/chapter_multilayer-perceptrons/index.html}{Link to chapter}

\section{Multilayer Perceptrons}
\begin{itemize}
    \item Incorporating Hidden Layers to network architecture
    \item Adding non linearity with activation functions
\end{itemize}

\section{Implementation of Multilayer Perceptrons from Scratch}
Similar to logistic/linear regression implementation involving
\begin{itemize}
    \item Define the model (net) and initialise params
    \item Define the loss
    \item Define the optimizer (stepping in the direction of neg gradient)
    \item Implement training loop (using efficient mini batch data loaders)
\end{itemize}


\section{Model Selection, Underfitting, and Overfitting}
\begin{itemize}
    \item here we go!
\end{itemize}

\vfill
\bibliography{../References}
\nocite{zhang2020dive}
\end{document}

