\documentclass[12pt,notitlepage]{article}

% for margins
\usepackage[a4paper,left=3cm,right=2cm,top=2.5cm,bottom=2.5cm]{geometry}
% for citations
\usepackage{apacite}
% for hyperlinks
\usepackage{hyperref}
% for maths symbols like the real numbers
\usepackage{amssymb}


\begin{document}
\bibliographystyle{apacite}

\title{\Large{\textbf{Chapter 5}}}
\date{October 9, 2020}
\author{Simon Ward-Jones\\simonwardjones16@gmail.com}

\maketitle
% \tableofcontents
\href{https://d2l.ai/chapter_deep-learning-computation/index.html}{Link to chapter}

\section{Deep Learning Computation}
\begin{itemize}
    \item A block could describe a single layer, a component consisting of multiple layers, or the entire model itself
    \item From a programing standpoint, a block is represented by a class
    \item Blocks take care of lots of housekeeping, including parameter initialization and backpropagation
    \item In pytorch params are collected from the class attributes
    \item Pytorch provides serialisation through state dict and torch.save and torch.load
\end{itemize}

\section{GPU}
\begin{itemize}
    \item In PyTorch, every array has a device, we often refer it as a context.
    \item Objects need to be on the same device to interact e.g. multiply/add
    \item Copying between devices is easy but slow so watch out
\end{itemize}

\vfill
\bibliography{../References}
\nocite{zhang2020dive}
\end{document}

